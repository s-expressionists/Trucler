\chapter{Augmenting the environment}

\section{Adding and annotating variables}

\subsection{Adding a lexical variable}

\Defun {add-lexical-variable} {client environment name \optional identity}

This function returns a new environment that is like
\textit{environment} except that it has been augumented with a lexical
variable named \textit{name}.  The optional argument \textit{identity}
can be supplied by client code to distinguish different lexical
variables with the same name.

\subsection{Adding a special variable}

\Defun {add-special-variable} {client environment name}

This function returns a new environment that is like
\textit{environment} except that it has been augumented with a special
variable named \textit{name}.

\section{Adding and annotating functions}

\subsection{Adding a local function}

\Defun {add-local-function} {client environment name \optional identity}

This function returns a new environment that is like
\textit{environment} except that it has been augumented with a local
function named \textit{name}.  The optional argument \textit{identity}
can be supplied by client code to distinguish different functions with
the same name.

\section{Adding a block}

\Defun {add-block} {client environment name \optional identity}

This function returns a new environment that is like
\textit{environment} except that it has been augumented with a block
named \textit{name}, which must be a symbol.  The optional argument
\textit{identity} can be supplied by client code to distinguish
different blocks with the same name.

\section{Adding a tag}

\Defun {add-tag} {client environment tag \optional identity}

This function returns a new environment that is like
\textit{environment} except that it has been augumented with a tag
named \textit{tag}, which must be a \emph{go tag}, i.e. a symbol or an
integer.  The optional argument \textit{identity} can be supplied by
client code to distinguish different tags with the same name.
